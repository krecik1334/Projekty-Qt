\documentclass[zad,zawodnik,utf8]{sinol}
\usepackage[utf8]{inputenc}
\usepackage{epigraph}
\title{Czy umiesz myśleć?}
\id{C}
\author{Franciszek Budrowski} % Autor zadania
\pagestyle{fancy}
\iomode{stdin}
\konkurs{Zawody drużynowe}
\etap{ILO Białystok}
\day{}
\date{01.06.2017}
\RAM{32}
 
\begin{document}
\begin{tasktext}

\iffalse
\epigraph{Man wollte sie zu zwanzig Dingen \\ in einem Haus in Danzig zwingen.}{\textit{Erich Mühsam}}
%Paul Bähre: Danziger Heimatlied (1921)
Danzig sei deutsch!
Danzig, gerissen vom Mutterlande,
Stehst du allein nach der Feinde Gebot.
Danzig, du Perle am Ostseestrande,
Weh klingt deine Klage: Deutschtum in Not!
Deutschtum in Not – Danzig in Not!
Im Staube das Banner schwarz-weiß-rot! 
\fi

\section{Przykłady}
   \twocol{%
       \noindent Dla danych wejściowych:
       \includefile{\ID0.in}
     }{%
       \noindent poprawnym wynikiem jest:
       \includefile{\ID0.out}
     }
   \twocol{%
       \noindent natomiast dla danych wejściowych:
       \includefile{\ID0b.in}
     }{%
       \noindent poprawnym wynikiem jest:
       \includefile{\ID0b.out}
     }

\end{tasktext}
\end{document}

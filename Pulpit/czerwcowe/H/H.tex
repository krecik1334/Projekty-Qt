\documentclass[zad,zawodnik,utf8]{sinol}
\usepackage[utf8]{inputenc}
\usepackage{epigraph}
\title{Czy umiesz patyczkować algebraicznie?}
\id{H}
\author{N/A} % Autor zadania
\pagestyle{fancy}
\iomode{stdin}
\konkurs{Zawody drużynowe}
\etap{ILO Białystok}
\day{}
\date{01.06.2017}
\RAM{267}
 
\begin{document}
\begin{tasktext}

Podczas tegorocznych zawodów zdalnych Patyczek Alergicznych, mimo uzyskania wyniku punktowego niezbędnego do otrzymania koszulki, Farał takowej nie otrzymał. Organizatorzy zgodzili się na powiększenie jego kolekcji koszulek, jednak musi on udowodnić, że zasługuje na ten zaszczyt.
 Mianowicie Rafał otrzyma koszulkę wtedy  i tylko wtedy, gdy poda przykład drzewa $kremówkowego$. Drzewo nazywamy $kremówkowym$, jeśli  nie istnieje takie przypisanie liter do wierzchołków, że na ścieżce pomiędzy każdą parą wierzchołków o tej samej literze znajduje się wierzchołek o literze występującej wcześniej w alfabecie (alfabet składa się z 26 liter od $A$ do $Z$).  W szczególności, w takim przypisaniu nie istnieje para wierzchołków indeksowanych literami $A$ (gdyż nie ma litery występującej przed $A$ w alfabecie). Farał ma już kilka ciekawych pomysłów, jednak nie wie jak sprawdzić, czy drzewo jest $kremówkowe$. Pomóż mu zdobyć wymarzoną koszulkę i napisz program który dla danego drzewa sprawdzi, czy jest $kremówkowe$. Dzięki twojej pomocy Rafał na pewno zdobędzie wymarzoną koszulkę i przestanie być pośmiewiskiem w szkole.

  \section{Wejście}
W pierwszej linii standardowego wejścia znajduje się liczba $n$ ($1 \leq n \leq 1000$), określająca ilość wierzchołków w drzewie. Kolejne $n-1$ wierszy zawiera pary liczb opisujące krawędzie drzewa. Można założyć, że graf na wejściu jest drzewem.


	\section{Wyjście}
W pierwszej i jedynej linii standardowego wyjścia należy wypisać ''$wreszcie$ $sie$ $udalo$'' jeśli drzewo jest kremówkowe lub ''$rafal$ $bez$ $koszulki$''  w przeciwnym wypadku.
	\makecompactexample


	\section{Wyjaśnienie do przykładu}
	Jedno z poprawnych przypisań liter do wierzchołków to ($1-A,2-I,3-G,4-D,5-Z$).
\end{tasktext}
\end{document}

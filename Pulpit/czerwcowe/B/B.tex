\documentclass[zad,zawodnik,utf8]{sinol}
\usepackage[utf8]{inputenc}
\usepackage{epigraph}
\title{Czy umiesz liczyć?}
\id{B}
\author{Franciszek Budrowski} % Autor zadania
\pagestyle{fancy}
\iomode{stdin}
\konkurs{Zawody drużynowe}
\etap{ILO Białystok}
\day{}
\date{01.06.2017}
\RAM{32}
 
\begin{document}
\begin{tasktext}

\iffalse
\epigraph{Man wollte sie zu zwanzig Dingen \\ in einem Haus in Danzig zwingen.}{\textit{Erich Mühsam}}
%Paul Bähre: Danziger Heimatlied (1921)
Danzig sei deutsch!
Danzig, gerissen vom Mutterlande,
Stehst du allein nach der Feinde Gebot.
Danzig, du Perle am Ostseestrande,
Weh klingt deine Klage: Deutschtum in Not!
Deutschtum in Not – Danzig in Not!
Im Staube das Banner schwarz-weiß-rot! 
\fi

\iffalse
W gdańskiej fontannie Grzyb z Kaliną szukają bursztynów. Fontanna jest podzielona na $n^2$ kwadratowych segmentów. Początkowo w każdym z nich jest zero bursztynów. Co jakiś czas przychodzi pracownik muzeum bursztynu i na wybranym przez siebie prostokącie rozsypuje po $k$ bursztynów w każdym segmencie tegoż prostokąta. Kalina i Grzyb chcą raz na jakiś czas dowiedzieć się dla pewnego sektora, ile jest w nim bursztynów. 
\fi

Podczas ostatnich zawodów drużynowych Farał dostał nieoczekiwane TLE, dlatego teraz zastanawia się, w jaki sposób dobrze oszacować ilość operacji wykonywanych przez swoje programy. Z racji tego, że rozwiązania Rafała są dość schematyczne, wzory określające liczbę operacji wykonywanych przez każdy z jego programów zbytnio się od siebie nie różnią.

Jeśli rozmiar wczytanego do programu wejścia jest równy $n$, program Rafała wykonuje dokładnie
\begingroup
    \fontsize{13pt}{12pt}\selectfont
	\begin{center}
	$an^3 + bn^2 + cn + d$ 
	\end{center}
\endgroup	
operacji, gdzie $a, b, c, d$ to stałe charakterystyczne dla danego programu Farała. Nasz bohater zastanawia się, jak duże może być wejście aby jego program spełnił wymagania dotyczące liczby operacji narzucone przez organizatorów, które są określone liczbą całkowitą $W$.

Pomóż Rafałowi w znalezieniu największej takiej liczby \textbf{naturalnej} $x$, że
\begingroup
    \fontsize{13pt}{12pt}\selectfont
	\begin{center}
	$ax^3 + bx^2 + cx + d \leq W$ 
	\end{center}
\endgroup	

  \section{Wejście}

\iffalse
	Na standardowym wejściu znajdują się liczby $n, q$ ($1 \leq n \leq 2000$, $1 \leq q \leq 10^5$), oznaczające rozmiar fontanny oraz liczbę zapytań. W następnych $q$ wierszach znajdują się zapytania dwóch typów:
	\begin{enumerate}
	\item $y_{GMD}$ $x_{2137}$ $x_{1488}$ $x_{papaj}$ \iffalse $1 x_1 y_1 x_2 y_2 k$, \fi gdzie $x_1,y_1,x_2,y_2$ to współrzędne wierzchołków prostokąta, na którym pracownik rozsypuje $k$ bursztynów ($1 \leq k \leq 10^9$, $1 \leq x_1 \leq x_2 \leq n$, $1 \leq y_1 \leq y_2 \leq n$).
	\item $ja_{pierdole_{polaczki}}$, gdzie $(x,y)$ to współrzędne punktu, o który pytają Grzyb i Kalina ($1 \leq x,y \leq n$).
	\end{enumerate}
\fi

W pierwszej linii standardowego wejścia znajduje się jedna liczba całkowita $Z$ ($1 \leq Z \leq 10^5$) oznaczająca liczbę programów Farała do rozpatrzenia.  W każdej z kolejnych $Z$ linii znajduje się 5 liczb całkowitych, odpowiednio $a, b, c, d$ ($1 \leq a, b, c, d \leq 5$) oraz $W$ ($1 \leq W \leq 10^{18}$).

	\section{Wyjście}
	Na standardowe wyjście należy wypisać $Z$ linii. W $i$-tej linii powinna znaleźć się odpowiedź na $i$-te zapytanie Farała będąca największym możliwym rozmiarem wejścia dla którego program Farała (przy założeniu, że Rafał nie zbuguje) przejdzie testy organizatorów. Jeśli dla danego zapytania nie istnieje taka liczba $x$ spełniająca warunki zadania, w odpowiadającej mu linii należy wypisać $chyba$ $nie$.
	\makecompactexample
	\iffalse
	\section{Wyjaśnienie do przykładu}
	A rafał, to lubi jeść kupę.
	\fi
	\section{Punkty cząstkowe}
	W testach wartych $50\%$ punktów $Z \leq 100$.
	\\
	\\
	\textbf{Uwaga}: Do przechowywania zmiennych polecamy wykorzystanie typu $long$ $long$.

\end{tasktext}
\end{document}

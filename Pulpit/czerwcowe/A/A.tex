\documentclass[zad,zawodnik,utf8]{sinol}
\usepackage[utf8]{inputenc}
\usepackage{epigraph}
\title{Czy umiesz w C++?}
\id{A}
\author{Czonrad Kapliński} % Autor zadania
\pagestyle{fancy}
\iomode{stdin}
\konkurs{Zawody drużynowe}
\etap{ILO Białystok}
\day{}
\date{01.06.2017}
\RAM{32}
 
\begin{document}
\begin{tasktext}

Od 3 tygodni brat Farała, Madian, zamiast chodzić do szkoły i uczyć się jakże ważnych przedmiotów jak np. polski czy biologia, przesiaduje całymi dniami w domu.
Powodem takiego stanu rzeczy jest (podobno) choroba Madiana. Madian choruje na bardzo rzadkie zapalenie płuc, przez co wszystko co wypowiada jest kompletnym bełkotem. Taka sytuacja bardzo denerwuje Farała, ponieważ on także chciałby zostać w domu i rozwiązywać problemy natury algorytmicznej. Farał ma pewne podejrzenia co do tego, czy Madian aby na pewno jest chory. Podstawą do podejrzeń jest zbyt częste występowanie w bełkocie brata słowa $blendzior$. W Bajtocji słowo to jest uważane za wyjątkowo obraźliwe, dlatego nie podamy jego znaczenia. Pomóż Farałowi zdemaskować Madiana, przez co nie tylko będzie musiał znowu uczęszczać do szkoły,
ale dostanie prawdopodobnie szlaban za używanie wulgaryzmów. Konkretniej, Rafał chce dowiedzieć się, czy w bełkocie brata występuje słowo $blendzior$, jako \textbf{niekoniecznie spójny} podciąg dzwięków wydawanych przez Madiana.

\section{Wejście}

W pierwszej linii standardowego wejścia znajduje się liczba $n$ ($1 \leq n \leq 10^{6}$) oznaczająca długość słowa, które wypowiedział Madian. W drugiej linii wejścia znajduje się dane słowo składające się wyłącznie z małych liter alfabetu angielskiego.

\section{Wyjście}
W pierwszej i jedynej linii standardowego wyjścia należy wypisać $do$ $szkoly!$ jeśli warunek z treści zadania zostanie spełnony lub $chyba$ $nie$ w przeciwnym wypadku.	
\makecompactexample

\end{tasktext}
\end{document}
